\documentclass[18pt]{beamer}
\usepackage[T1]{fontenc}
\usepackage[protrusion=true,expansion=true]{microtype}
%usepackage[minionint,mathlf]{MinionPro}
%\usepackage[sc,osf]{mathpazo}
\usepackage{listings}
\usepackage{diagrams}
%\renewcommand{\sfdefault}{Myriad-LF}

\newcommand{\fto}{\;\Rightarrow\;}
\definecolor{gray_ulisses}{gray}{0.55}
\definecolor{castanho_ulisses}{rgb}{0.71,0.33,0.14}
\definecolor{preto_ulisses}{rgb}{0.41,0.20,0.04}
\definecolor{green_ulises}{rgb}{0.2,0.75,0}

\lstdefinelanguage{HaskellUlisses} {
	basicstyle=\ttfamily,
	sensitive=true,
	morecomment=[l][\color{gray_ulisses}\ttfamily]{--},
	morecomment=[s][\color{gray_ulisses}\ttfamily]{\{-}{-\}},
	morestring=[b]",
	stringstyle=\color{red},
	showstringspaces=false,
	%numberstyle=,
	numberblanklines=true,
	showspaces=false,
	breaklines=true,
	showtabs=false,
	emph=
	{[1]
		FilePath,IOError,abs,acos,acosh,all,and,any,appendFile,approxRational,asTypeOn,
		asinh,atan,atan2,atanh,basicIORun,break,catch,ceiling,chr,compare,concat,concatMap,
		const,cos,cosh,curry,cycle,decodeFloat,denominator,digitToInt,div,divMod,drop,
		dropWhile,either,elem,encodeFloat,enumFrom,enumFromThen,enumFromThenTo,enumFromTo,
		error,even,exp,exponent,fail,filter,flip,floatDigits,floatRadix,floatRange,floor,
		fmap,foldl,foldl1,foldr,foldr1,fromDouble,fromEnum,fromInt,fromInteger,fromIntegral,
		fromRational,fst,gcd,getChar,getContents,getLine,head,id,inRange,index,init,intToDigit,
		interact,ioError,isAlpha,isAlphaNum,isAscii,isControl,isDenormalized,isDigit,isHexDigit,
		isIEEE,isInfinite,isLower,isNaN,isNegativeZero,isOctDigit,isPrint,isSpace,isUpper,iterate,
		last,lcm,length,lex,lexDigits,lexLitChar,lines,log,logBase,lookup,map,mapM,mapM_,max,
		maxBound,maximum,maybe,min,minBound,minimum,mod,negate,not,notElem,null,numerator,odd,
		or,ord,otherwise,pi,pred,primExitWith,print,product,properFraction,putChar,putStr,putStrLn,quot,
		quotRem,range,rangeSize,read,readDec,readFile,readFloat,readHex,readIO,readInt,readList,readLitChar,
		readLn,readOct,readParen,readSigned,reads,readsPrec,realToFrac,recip,rem,repeat,replicate,return,
		reverse,round,scaleFloat,scanl,scanl1,scanr,scanr1,seq,sequence,sequence_,show,showChar,showInt,
		showList,showLitChar,showParen,showSigned,showString,shows,showsPrec,significand,signum,sin,
		sinh,snd,span,splitAt,sqrt,subtract,succ,sum,tail,take,takeWhile,tan,tanh,threadToIOResult,toEnum,
		toInt,toInteger,toLower,toRational,toUpper,truncate,uncurry,undefined,unlines,until,unwords,unzip,
		unzip3,userError,words,writeFile,zip,zip3,zipWith,zipWith3,listArray,doParse
	},
	emphstyle={[1]\color{blue}},
	emph=
	{[2]
		Bool,Char,Double,Either,Float,IO,Integer,Int,Maybe,Ordering,Rational,Ratio,ReadS,ShowS,String,
		Word8,InPacket
	},
	emphstyle={[2]\color{castanho_ulisses}},
	emph=
	{[3]
		case,class,data,deriving,do,else,if,import,in,infixl,infixr,instance,let,
		module,of,primitive,then,type,where
	},
	emphstyle={[3]\color{preto_ulisses}\textbf},
	emph=
	{[4]
		quot,rem,div,mod,elem,notElem,seq
	},
	emphstyle={[4]\color{castanho_ulisses}\textbf},
	emph=
	{[5]
		EQ,False,GT,Just,LT,Left,Nothing,Right,True,Show,Eq,Ord,Num
	},
	emphstyle={[5]\color{preto_ulisses}\textbf}
}

\lstnewenvironment{code}
{\hrulefill \lstset{language=HaskellUlisses}}{}

\usepackage{textcomp}
\usepackage{graphicx}
\usepackage{semantic}

\usetheme{default}
\usefonttheme{professionalfonts}
%\usefonttheme[]{structuresmallcapsserif}
\begin{document}

\frame{
  \begin{center}
    Applicative programming\\
    with effects\\
  \end{center}
}

\begin{frame}[fragile]
  The Functor definition\\

  \begin{code}
    class Functor f where
      fmap :: (a -> b) -> f a -> f b
  \end{code}

\end{frame}

\begin{frame}[fragile] \frametitle{Category}
  A Category is:\\
  \begin{gather*}
    \text{A collection of objects:} \quad A, B, C, \dotsc\\
    \text{A collection of morphisms:} \quad f, g, h, \dotsc\\
  \end{gather*}
  Morphisms lies between objects: $f \colon A \to B$,\\
  For any object $A$, a morphism $Id_A \colon A \to A$,\\
  For any two morphisms, $f \colon A \to B, \; g \colon B \to C$ a
  composition $g \circ f \colon A \to C$\\
  \begin{diagram}
    A & \rTo^{f} & B & \rTo^{g} & C
  \end{diagram}
\end{frame}

\begin{frame}[fragile] \frametitle{Example}
  A Directed graph is a category:\\
  Objects are the vertices $x, y, z \in V$\\
  If there is a edge $e \in E$ with $s(e) = x$ and $t(e) = y$ then $e
  \colon x \to y$ is a morphism.\\
  The identity morphism is vacously true.\\
  Paths play the role of compositions.\\--\\

  \emph{Other Examples:} vector spaces, monoids, groups, topological
  spaces, etc.
\end{frame}

\begin{frame}[fragile] \frametitle{Functor}
  If $\mathcal{C}$ and $\mathcal{D}$ are categories, then $F \colon
  \mathcal{C} \fto \mathcal{D}$ iff:
  \begin{itemize}
  \item $F$ maps an object $A$ to $F A$
  \item $F$ maps a morphism $f \colon A \to B$ to $F f \colon F A \to
    F B$
  \item For any object $A$, $F \; Id_A = Id_{F A}$
  \item For all morphisms $f, g$: \quad $F (g \circ f) = F g \circ F f$
  \end{itemize}
  $F$ is then called a \emph{functor}.
\end{frame}

\begin{frame}[fragile] \frametitle{Example}
  The (standard) instance of a functor\\
  \begin{code}
    class Functor [] where
      fmap f [] = []
      fmap f (x : xs) = f x : (fmap f xs)
  \end{code}
\end{frame}

\begin{frame}[fragile] \frametitle{Leading example}
What we have without Applicative Functors:\\

\begin{code}
data Exp v = Var v
           | Val Int
           | Add (Exp v) (Exp v)

eval :: Exp v -> Env v -> Int
eval (Var x) g = fetch x g
eval (Val i) g = i
eval (Add p q) g = eval p g + eval q g
\end{code}
Not very pretty - nor general
\end{frame}

\begin{frame}[fragile] \frametitle{S, K combinators}
We can do better:\\
\begin{code}
eval :: Exp v -> Env v -> Int
eval (Var x) = fetch x
eval (Val i) = K i
eval (Add p q) = K (+) 'S' eval p 'S'eval q
\end{code}
where\\
\begin{code}
K :: a -> env -> a
K x g = x

S :: (env -> a -> b)
         -> (env -> a) -> (env -> b)
S ef es g = (ef g) (es g)
\end{code}
\end{frame}

\begin{frame}[fragile]
Better - not good, not general.
\end{frame}

\begin{frame}[fragile] \frametitle{Applicative Functor}

\begin{align*}
  \mathbf{infixl} & \; 4 \; \circledast\\
  \mathbf{class} & \; \mathrm{Applicative} \; f \; \mathbf{where}\\
  \quad & \mathrm{pure} \; :: \; a \rightarrow f a\\
  \quad & (\circledast) \; :: \; f (a \rightarrow b) \rightarrow f a
\rightarrow f b
\end{align*}
Rules:
\begin{align*}
  \mathbf{Identity:} \quad & \mathrm{pure} \; \mathrm{id} \;
  \circledast u = u\\
  \mathbf{Composition:} \quad & \mathrm{pure} \; (\cdot) \circledast u
  \circledast v \circledast w = u \circledast (v \circledast w)\\
  \mathbf{Homomorphism:} \quad & \mathrm{pure} \; f \circledast x =
  \mathrm{pure} \; (\lambda f \rightarrow f x) \circledast f\\
  \mathbf{Interchange:} \quad & u \circledast x = \mathrm{pure} \;
  (\lambda f \rightarrow f x) \circledast f\\
\end{align*}
%Dollar sign
\end{frame}

\begin{frame}[fragile]
Canonical form:\\
pure $f \circledast u_1 \circledast \ldots \circledast u_n \equiv |[f u_1 \ldots u_n|]$\\
so in practice\\
\begin{code}
pure f <*> u1 <*> ... <*> un
\end{code}
\end{frame}

\begin{frame}[fragile]
Let's return to our start example - but first!\\
\begin{code}
instance Applicative ((->) env) where
pure x = \g -> x --K
ef <*> ex = \g -> (ef g)(ex g) -- S
\end{code}

So now:
\begin{code}
eval :: Exp v -> Env v -> Int
eval (Var x) = fetch x
eval (Val i) = pure i
eval (Add p q) = pure (+) <*>  (eval p) <*> (eval q)
\end{code}
Much better.

\end{frame}

\begin{frame}[fragile]
Can now make a aplicative distributor for lists:\\
\begin{align*}
&dist :: Applicative f &=>& [f a] -> f [a]\\
&dist [] &=& |[ [] |]\\
&dist (v : vs) &=&  |[ (:) v (dist vs) |]
\end{align*}

Example:\\
\begin{code}
flakeyMap :: (a -> Maybe b) -> [a] -> Maybe [b]
flakeyMap f ss = dist (fmap f ss)
\end{code}
\end{frame}

\begin{frame}[fragile]
Better choice:

\begin{align*}
&traverse :: Applicative f &=> (a -> f b) -> [a] -> f [b]
&traverse f [] &= |[ [] |]
&traverse f (x :xs) &= |[(:)(f x)(traverse f xs)|]
\end{align*}

\begin{code}
class Traversable t where
traverse :: APplicative f => (a -> f b) -> t a -> f (t b)
dist  :: Applicative f => t (f a) -> r (t a)
dist = traverse id           
\end{code}
Through this and the type system we can now create traversable functions
\end{frame}

\begin{frame}[fragile]\frametitle{Monoids and traversability}
\begin{align*}
\mathbf{class} \; \mathrm{Monoid} \; o \; \mathbf{where}\\
\emptyset &::& o\\
(\oplus) &::& o -> o -> o
\end{align*}  \\
These induce an applicative functor:\\

$\mathbf{newtype} \; \mathrm{Accy} \; o\; a = \mathrm{Acc} \{\mathrm{acc} \; :: \;o \}$

\begin{align}
&\mathbf{instance} \; \mathrm{Monoid} \; o &=>& \mathrm{Applicative} (\mathrm{Accy}\; o) \; \mathbf{where}\\
&\mathrm{pure} \_ &=& \mathrm{Acc} \; \emptyset\\
&\mathrm{Acc} \; o_1 \circledast \mathrm{Acc} o_2 &=& \mathrm{Acc}(o_1 \oplus o_2)
\end{align}
\end{frame}

\begin{frame}[fragile] \frametitle{}

accumulate :: (Traversable t, Monoid o) $=>$ (a $->$ o) $->$ t a $->$ o
accumulate f = acc $\cdot$ traverse (Acc $\cdot$ f)

newtype Mighty = Might\{might :: Bool \}
\begin{align*}
\mathbf{instance} \; &\mathrm{Monoid \;  Mighty} \; \mathbf{where}\\
&\emptyset &= \mathrm{Might} \; \mathrm{False}\\
&\mathrm{Might}\; x \oplus \mathrm{Might}\; y &= \mathrm{Might} (x \lor y)
\end{align*}

any :: Traversable t $=>$ (a $->$ Bool) $->$ t a $->$ Bool\\
any p = might $\cdot$ accumulate (Might $\cdot$ p)
\end{frame}
\end{document}

%%% Local Variables: 
%%% mode: latex
%%% TeX-master: t
%%% End: 
