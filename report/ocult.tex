
We used considerable time on the langauge of O'cult while attempting
to write a code generator from O'cult to 2d. In this section we
present the work done on this.

In order to understand the languages of O'cult and 2d, we wrote down
its operational semantics. The tool of choice for getting the
semantics type checked was the logical framework Twelf. The
development corresponds to a typed prolog written in the LF system, a
dependently typed variant of the lambda calculus. This was necessary
as O'cult has a rather evil evaluation rule on ``juxtaposition''
applications.

We then choose to eliminate the single odd rule from O'cult, namely
the rule ``Advice is least heeded where it is most needed''. This rule
eliminated, O'cult is a simple term rewriting language. A list of
rules are tried in succession. Each rule is a pattern match. If the
pattern matches, a replacement-term describes how the rewriting is to
be done.

The idea was to encode enough in 2d (eventually by hand) to be able to
store variable bindings and operating on these. Then rules could be
compiled such that if a rule failed to match then the next rule would
be tried. If the rule did match on the other hand, we would call the
appropriate replacer and then use the program itself, effectively
tail-calling to the next term-rewrite.

It became apparent however, as time progressed, that we would not be
able to complete this task. We initally estimated that we needed some
$3000$ to $3500$ lines of source code to complete such a work. Hence
we abandoned the idea of compiling O'cult and looked at a simpler
language.

In hindsight, we wasted quite some time on O'cult. But we envisioned
the SEL language we then created as a work towards a middle-ground
between 2d and O'cult. The idea was to construct a language extending
2d with the primitives we needed to compile O'cult. The SEL is work
towards this goal.

%%% Local Variables: 
%%% mode: latex
%%% TeX-master: "master"
%%% End: 
