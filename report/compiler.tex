\section{Compiler}
\label{compiler}

\subsection{Problems faced}
\label{compiler:problems}

\subsubsection{Layout}
\label{compiler:layout}
The main challenge of writing a compiler from our language to the 2d language, was making a system for layout out a valid 2d program. 

\subsubsection{Placement of wires}
\label{compiler:placement}


\subsection{Modules}
\label{compiler:modules}
In order to cut down on the needed work, we decided to encapsulate many of our functions in predefined modules. An example of this could be the multiplication operator, which we choose to handwriting, since we had no real way of automatically translating the language from a description to the 2d code \fixme{I am unsure if this is true - and if it is, whether how to describe it probably}. Even though we had already specified much of the logic needed for unary multiplication, the construction was not straightforward, as many of the particulars of the 2d language presented itself. We found the biggetst problem to be flow-control, as this required many wires, even for small cases, which made the program much harder to understand.

