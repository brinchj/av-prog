\chapter*{Task 3: 2D code generator}
\label{compiler}

\paragraph{Problems faced}
\label{compiler:problems}

\paragraph{Layout}
\label{compiler:layout}
The main challenge of writing a compiler from the SEL to the 2d
language, was making a layout system that would output a valid 2d program. 

\paragraph{Placement of wires}
\label{compiler:placement}

\paragraph{Modules}
\label{compiler:modules}
In order to cut down on the needed work, we decided to encapsulate
many of our functions in predefined modules. An example of this could
be the multiplication operator, which we attempted to write manually,
both to get a sense of how 2d worked, as well as optimizing our use of
time until the layout system was up and running. Even though we had
already specified much of the logic needed for unary multiplication,
the construction was not straightforward. We found the biggetst
problem to be flow-control, as this required many wires, even for
small cases, which made the program much harder to write and modify.\\

We therefore abondoned this approch, even for basic features of our language, and instead began to use our layouter to create the needed modules.\\
