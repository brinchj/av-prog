
\section{Compiling SEL to 2d}
\label{sec:compiling-sel-2d}

In this section, we present the methodology behind compiling from SEL
into 2d. The basic idea is that we define a representation of natural
numbers in 2d and then we use boxes of 2d to represent SEL programs in
2d. We begin by noting that most of our operations are straightforward
to compile into 2d code. The hard part is how to handle the boxes.

We let a box be a command and a set of wires, one for each direction
NESW. When creating a box, we just create 4 wires for the box. To
interconnect two boxes, we just have to set the wire of one box to the
wire of the other. We prune unneeded wires in the layout code later
on. This means we construct a graph of boxes, and then render it with
the layouter, described below.

Compiling SEL to 2d requires us to construct the right boxes and wire
them together in the correct way. We compile each SEL function
separately into its own module. When compiling a SEL expression, we do
it inside a state monad. The state in the monad tracks what the next
free wire-number is, the currently operating environment, what wires
have left the module and a list of input wires. The monad allows us to
quickly create new wires and eliminates the need for passing around a
large amount of boring information. We note that some of this
information is deemed static and as such the state monad could have
been wrapped in a reader transformer. But it was seen as too
complicated for a problem this simple.

A set of helper functions allows us to join boxes together in groups
in various ways. When we join the boxes into groups, we identify a
public interface to the group in each direction. The public interface
simplifies later join operations.


\paragraph{Compiling Expressions}
\label{sec:compiling-numbers}

Natural numbers are compiled like it was laid out in the hints PDF to
task 3. Zero is represented as one injection in the sum-type, and Succ
x is represented as the other. Expressions that use arithmetic makes
use of the standard library and simply calls the appropriate module.

Looking up in the environment compiles the tree in the environment and
sends its result to the place in the AST where it is needed. Finally,
a function call compiles its single input and then uses the
appropriate module.

\paragraph{Compiling the zeroCheck}
\label{sec:compiling-zerochk}

Compiling the zeroCheck is not straightforward. We link the check
into a case box. The Inr case is then then linked to the first case's
North side, ensuring that it is only executed if the value is unary
zero (in this case the function called must take care to ensure that
the input from the North is corrected), while the Inl case is inserted
in the environment and then sent to the North side of the module
calling the second expression.

The problem with this construction is that we have no way to merge
multiple outputs into one. Our solution will not work in the general
case and is inadequate. The problem of merging outputs in 2D is
solvable by using multiple module outputs, but this requires another
compilation strategy (see below).

\paragraph{Compiling functions}
\label{sec:compiling-functions}

Compiling a function has the complication that there is only a single
wire of input, but the input might be used multiple times in the
function body. We solve this problem by first counting the number of
occurrences in the body. We note that for the few cases we have there
is no problem in handling this. For a greater number of cases, one could
have utilized the generics of Haskell here.

We then construct a series of boxes, duplicating the input by
splitting tuples. The wires with the input is gathered and when
compiling expressions we keep these input wires in a list in the state
of the monad.

For the outputs, we track the wires currently outputted from the
module. This tracking is done with a list inside a state monad. Upon
having created the module, we then know exactly what wires exit the
module.

%%% Local Variables:
%%% mode: latex
%%% TeX-master: "master"
%%% End:
