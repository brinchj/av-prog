
\section{Compiling SEL to 2d}
\label{sec:compiling-sel-2d}

In this section, we present the methodology behind compiling from SEL
into 2d. The basic idea is that we define a representation of natural
numbers in 2d and then we use boxes of 2d to represent SEL programs in
2d. We begin by noting that most of our operations are straightforward
to compile into 2d code. The hard part is how to handle the boxes.

Recall that a box contains wires in each direction, NESW. To
interconnect two boxes, we just have to set the wire of one box to the
wire of the other. We prune unneeded wires in the layout code later
on.

We compile each SEL function seperately into its own module. When
compiling a SEL expression, we do it inside a state monad. The state
in the monad tracks what the next free wire-number is, the currently
operating environment, what wires have left the module and a list of
input wires. The monad allows us to quickly create new wires and
eliminates the need for passing around a large amount of boring
information. We note that some of this information is deemed static
and as such the state monad could have been wrapped in a reader
transformer. But it was seen as too complicated for a problem this
simple.

A set of helper functions allows us to join boxes together in groups
in various ways. When we join the boxes into groups, we identify a
public interface to the group in each direction. The public interface
simplifies later join operations.


\paragraph{Compiling Expressions}
\label{sec:compiling-numbers}

Natural numbers are compiled like it was laid out in the hints PDF to
task 3. Zero is represented as one injection in the sum-type, and Succ
x is represented as the other. Expressions that use arithmetic makes
use of the standard library and simply calls the appropriate module.

Looking up in the environment compiles the tree in the environment and
sends its result to the place in the AST where it is needed. Finally,
a function call compiles its single input and then uses the
appropriate module.

\paragraph{Compiling the zeroCheck}
\label{sec:compiling-zerochk}

\fixme{Write this}
\fixme{Change the code here as well}

\paragraph{Compiling functions}
\label{sec:compiling-functions}

Compiling a function has the complication that there is only a single
wire of input, but the input might be used multiple times in the
function body. We solve this problem by first counting the number of
occurrences in the body. We note that for the few cases we have there
is no problem in handling this. For a greater number of cases, one could
have utilized the generics of Haskell here.

We then construct a series of boxes, duplicating the input by
splitting tuples. The wires with the input is gathered and when
compiling expressions we keep these input wires in a list in the state
of the monad.

\fixme{Handling multiple outputs}

\section{Compiler Alternative}
\label{chap:compiler-alter}

Late in the project, due to the problems with compiling the zeroCheck,
we discovered a smarter compilation scheme of SEL to 2d. Note that
a module in 2d has continuation semantics. A use of a module define a
continuation point where the program will continue once the program
returns. Thus, one can CPS-transfrom SEL and output each part of the
CPS transformed program. It turns out this is surprisingly simple
though it uses modules heavily.

\begin{figure}
  \begin{center}
\begin{verbatim}
,...................................,
: zerocont3214                      :
: *===============*                 :
: !use "test3251" !                 :
: *===============*                 :
:        v                          :
: *============*  *================*:
: !case[N] E S !->!use "zcase3526" !-
: *============*  *================*:
:      	 |        *================*:
:        +------->!use "scase6543" !-
:                 *================*:
,...................................,
\end{verbatim}
  \end{center}
  \caption{Case compilation example}
  \label{fig:2}
\end{figure}

As an example, consider the compilation of a zero control. In Figure
\ref{fig:2}, we present the continuation-compilation of the zero
control. The Monad we need for this contain a writer transformer so we
can output modules as we go (\texttt{MonadState s (WriterT w m)}). The
control is compiled as a hardcoded module where the values of the
``uses'' can be replaced. The pseudocode for compilation is then:
\begin{verbatim}
compileE (ZCont test zcase scase) =
  do
    test_id <- compileE test
    zcase_id <- compileE zcase
    scase_id <- compileE scase
    zerocont_id <- new
    tell $ mkZerocont zerocont_id test_id zcase_id scase_id
    return zerocont_id
\end{verbatim}
where the \texttt{mkZerocont} function outputs the hardcoded box with
the identifiers substituted.

We believe this compilation scheme is much easier to implement that
the one we tried to implement. The key observation is that
continuation passing style fits 2d directly. We believe that all other
SEL-constructs are as easy to compile.

Unfortunately time prevents us from implementing this idea even though
we estimate it would take a couple of days at most. We also estimate
that this can be done in less than 150 lines of code. Comparing this
to our current implementation of above 1000 lines, this is a gain of
considerable size.

\paragraph{Optimization}
\label{sec:optimization}

In the above example, one does not even need to take care of 2d layout
at all. One can simply output all the modules. Optimization is
possible with a layouter however. One can easily implement an
optimizer inlining modules if the module always returns to the same
continuation. This optimization will reify as large modules as
possible and is fairly easy to program.

%%% Local Variables:
%%% mode: latex
%%% TeX-master: "master"
%%% End:
