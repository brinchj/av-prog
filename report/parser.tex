\label{parser}
In order to translate the code into 2d, we must first parse the Simple
Expression Language (SEL) and turn it into an abstract syntax
tree. This will be done in an independent module in order to maximize
the modularity of the code.

\paragraph{Use of Parsec}
Due to its age and stability we have chosen to use the Parsec Haskell
module to write our Parser. Even though it might be possible to create
something better through the use of Monad transformers as described in
\cite{partial+parsing}, this might take a long time, as well as
introducing errors that already have been found and eliminated in
Parsec.

\paragraph{Experiences with Parsec}
While parsec as a rule has been usable, there have still been areas that have given us a great deal of problems. A
good example of this is when left recursion is required, where one
needs to use the built-in expressionBuilder to solve the problem. More
problematic were some of the errors we originally tried to implement
the parser for O'cult, which included the expression ``e ::= e e''.\\

\paragraph{Grammar}
The syntax of our language can be found in its native form in our
Parse.hs file. However, for completeness we have included it here. In
the following text written like \bnf{so} is terminating, while
everything else is not:\\

expr ::= zeroCont $\mid$ \bnf{z} $\mid$ succ $\mid$ parens op $\mid$ lookUp $\mid$ call $\mid$ op\\

program ::= function $\mid$ function functions\\

function ::= name \bnf{=>} constStart op\\

functions ::= \bnf{,} function functions $\mid$ $\epsilon$\\

constStart \bnf{[} consts \bnf{]}\\

consts ::= const \bnf{,} consts $\mid$ $\epsilon$\\

const ::= name \bnf{=} op\\

zeroCont ::= \bnf{zeroCheck} \bnf{(} expr \bnf{)} \bnf{\{} expr \bnf{\}} \bnf{else} \bnf{\{} expr \bnf{\}}\\

call ::= \bnf{Call} name op\\

lookUp ::= \bnf{Lookup} name\\

op ::= expr sign op | expr\\

sign ::= \bnf{+} | \bnf{*}\\

parens ::=\bnf{(} expr \bnf{)}\\

succ ::= \bnf{s} op\\

name ::= [a-zA-Z]+

\paragraph{Interpreter}
During the creation of the parser we wrote an interpreter in
Haskell. We did this for the twofold purpose of
\begin{enumerate}
\item Checking whether the parser parsed the language correctly
\item Gain a better understanding on how to do to the various
  operations on unary values. This proved to be especially beneficial
  for multiplication.
\end{enumerate}
Likewise we made an Abstract Syntax Tree (AST) printer, in order to
check the result of both the final and intermediate stages. We
likewise made a printer for the environment and the functions.

%%% Local Variables:
%%% mode: latex
%%% TeX-master: "master"
%%% End:
