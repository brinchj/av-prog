\subsection{Language features}

\paragraph{Variables}
We have added variables to our language, so the user can specify any
(reasonable) number of variables, which can then be looked up later in
the program. We have limited the creation of Variables to the start of
a function, as well made the restriction that the user cannot change
the value of the variables at runtime.

In the current version the values are stored as a AST, which have to
be evaluated when they are needed. This means that the values not only
can be constants, but it can also be plus and multiply expressions
(and indeed contain other variables and function calls). The variables
are stored in a simple list.

\paragraph{Function calls}
We have expanded our language to allow the use of functions that can
be called from anywhere in the code. Like variables, the functions are
stored in a list, which is searched through each time there is a
function call. A function takes exactly one input, and returns exactly
one output.

\paragraph{ZeroControl}
We have implemented flow control in our language in the form of a
zeroCont. Given a condition and 2 expressions, zeroCont first checks
whether the condition is zero or the successor of some $k$. If the
condition evaluates to zero the first expression is executed. If the
condition is the successor of some $k$ the second expression is
evaluated - with a new variable list, where the value of $k$ has been
added as the first element in the variable list (under the variable
name ``k'').

\paragraph{Expressiveness of the language}
Our language is expressive enough to describe the factorial function,
since we can call functions recursively, have flow control, as well as
basic arithmetic operations such as multiply.

\paragraph{Possible further expansions}
If the language had to be further expanded, to allow it to calculate
more elaborate functions, we would first start by allowing variables
being added in the code itself (and not only at the start of the
function). Furthermore it would be wise to allow functions to take
multiple arguments, since it would both increase the scope of what
could be calculated by the program, but also the convenience - though
this should be tempered by the need to be able to compile it to
2d.

\subsection{SEL Interpreter}
In order to test the SEL parser we implemented an SEL interpreter
that interprets the AST returned by the parser. This is implemented
in the function \texttt{eval} that given an AST together with an
environment, evaluates and returns the result.

\begin{samepage}
  \begin{exmp}
    The AST representing the factorial function applied to $3$:
\begin{verbatim}
  Main => [] (Call Fact (s s s z)),
  Fact => [] (zeroCheck(Lookup input) {(s z)}
               else {(Lookup input) * (Call Fact (Lookup k))}
\end{verbatim}
    Evaluates to:
\begin{verbatim}
  (s s s s s s z)
\end{verbatim}
    Showing that the faculty of $3$ evaluates to $6$ (as expected).
  \end{exmp}
\end{samepage}


%%% Local Variables:
%%% mode: latex
%%% TeX-master: "master"
%%% End:
