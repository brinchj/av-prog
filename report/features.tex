\section{Language}


\subsection{Language features}
\subsubsection{Variables}
We have added variables to our language, so that the user can specify any (reasonable) number of variables, which then can be looked up later in the program. We have limited the creation of Variables to the start of the function, as well made the restriction that the user cannot change the value of the values at runtime.

In the current version the values are stored as a Abstract Syntax Tree, which has to be evaluated, the values can also be plus and multiply expressions (and indeed contain other variables. These Abstract Syntax trees are evaluated at runtime - and reevaluated each time the variable is looked up. The Variables are stored in a list - as we found this to be the easiest to get up and running - if it should be used for production, it would be obvious to use a sort of hash-map (with proper handling of collisions, of course).

\subsubsection{Function calls}

We have expanded our language to allow the use of functions, that can be called from anywhere in the code. Like variables, the functions are stored in a list, which is searched through each time there is a function call. A function takes exactly one input, and returns exactly one output.

\subsubsection{Zero }
We have implemented flow control in our language in the form of a zeroCont. Given a condition and 2 expressions, zeroCont first checks whether the condition is zero or the successor of some k. If the condition evaluates to zero then the first expression is executed. If the condition is the successor of some k, then the second expression is evaluated - with a new environment, where the value of k (under the variable name k) has been added as the first element in the environment list.

\subsection{Expressiveness of the language}
Our language is expressive enough to describe the faculty function, since we can call functions recursively, has flow control, as well as basic arithmetic operations such as multiply. 


\subsection{Possible further expansions}
If the language had to be further expanded, to allow it to calculate more elaborate functions. We would first start by allowing variables being added in the code itself, and not only at the start of the function. Furthermore it would be wise to allow functions to take multiple arguments, since it would both increase the scope of what could be calculated by the program, but also the convenience - with the obvious caveat that one should be able to compile it 2d, of course.
