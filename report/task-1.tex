In this task, we implement the Universal Machine from the 2006 ICFP
contest in Haskell. The file \cite{um+spec} contains the specification
of the machine which are to be implemented. We can readily observe
that this machine is simple once the wording of the specification has
been decoded: it is a register machine with a store of
arrays. The instruction set is minimal at 13 instructions. It is pretty
clever as many instructions are chosen such that they can be used for
multiple purposes in the machine.

To implement the UM, one needs four pieces in the puzzle:
\begin{itemize}
\item Read in the UM program from disk as a byte stream. Convert it to
  opcodes (unsigned 32-bit words)
\item Provide an implementation of the store of the machine.
\item Provide an implementation of the register set of the machine.
\item Provide a proper interpreter. The interpreter loads the next
  instruction and dispatches it for execution before looping again.
\end{itemize}
Of these, the latter two has to be reasonably fast. One can expect the
interpreter loop to get billions of instruction executions. Hence it
makes sense to make this loop as tight and small as possible. Also, we
hypothesized that the store needed to be fast.

\paragraph{Startup}
\label{sec:startup}

Starting the machine is rather straightforward in Haskell. We load the
UM opcodes from a file and chop four bytes up into a \texttt{Word32}
which is the unsigned integer type most of our system works with. The
store is then initialized with this program and the main interpreter
is entered. When the program halts, we exit the interpreter.

Since this is Haskell, most of the top-level program lives inside the
IO monad to give us access to file I/O etc. The interpreter itself
lives inside the IO monad too, to allow standard input and output while
running the UM.

\paragraph{Store}
\label{sec:store}

The machine store underwent several iterations before the final version. We
knew from the start the store was important, so we created a type
class \texttt{State} for stores. The type class supports a simple
set of values suitable sparse for implementation of the register-to-memory
operations in the UM.

Our first implementation was list-based in a peculiar way. We stored
triples $(arr, idx, v)$ noting that the array $arr$ at location $idx$
had the value $v$. Lookup was done by linear search through this
list. This gave us a quick storage implementation that could be used
for testing, while the interpreter was being developed. Of course,
this was too slow for the full UM and quickly resulted in an overflow
of the stack.

The second iteration changed to using \texttt{Data.Sequence} from the
Haskell Standard Library. Sequences are finger-tree based so we could
expect $\OO{\lg n}$ lookup performance. We use data-structural
bootstrapping \cite{okasaki+pfds} to provide two-dimensional arrays
where the size of one dimension may differ in length. Each array
element is a \texttt{Seq Word32} which is then wrapped up inside
another \texttt{Seq}. A separate counter keeps track of the next
allocation index for malloc. Sequences in Haskell are always indexed by signed
integers. We convert to and from signed/unsigned integers using the
\texttt{fromIntegral} converter.\\

The next iteration provided correct freeing of storage. The counter
was converted to a list keeping track of the free array segments. If
no element is on the free-list, we create a new one. This cut memory
usage down.

However, the performance of the UM was still not satisfactionary, which
necessitated another iteration. The store was living inside the \texttt{Maybe}
monad at this point to keep track of errors. This was first changed to the
\texttt{IO} monad and errors was made fatal as in thread with the UM
specification \cite{um+spec}. Then the inner arrays were changed to
unboxed arrays of unsigned integers: \texttt{IOUArray Word32 Word32}
which in theory would improve the run-time from $\OO{\lg n} + \OO{\lg
  m}$ to $\OO{\lg n}$, shaving off the last lookup. Finally, the
$0$-array, containing the program was moved to its own tuple-slot such
that the often occurring opcode lookup could run in constant
time. This yielded the final type of the store as:
\begin{verbatim}
type Arr = IOUArray Word32 Word32
type Store = ([Word32], Arr, Seq (Arr))
\end{verbatim}

\paragraph{Interpreter}
\label{sec:interpreter}

The interpreter consists of an instruction decoder and the
step-loop. The instruction decoder takes an unsigned integer and
decodes the instruction to an internal algebraic datatype. This
internal representation is then dispatched. Great care has been taken
to decode the registers into meaningful names so the interpreter would
become easier to maintain and lower the margin for errors. Rather than
passing around a tuple \texttt{(Word32, Word32, Word32)}, we use a
named record to tag each register.

The step-loop tail-calls itself with an updated state. It tracks the
store, the register set and the program counter. Since we needed the
ability to do I/O, we decided to put the step-loop inside the
\texttt{IO} monad. It is possible to avoid the \texttt{IO}
monad, though it was deemed too complicated. The input can be
converted to a character stream of infinite length. The output can be
controlled with a \texttt{Writer} monad. Finally, one can use the
right kind of \texttt{ST} monad to track the store. These monads are
then stacked by use of Monad Transformers \cite{monad+transformer}.

The rest of the interpreter step-loop is straightforward, since all
the advanced operations are accounted for in either the store or the
register set.

\paragraph{Registers}
\label{sec:registers}

Originally, we just used a store as the register-set with a single
array containing 8 slots. In the course of the development, and with
the advent of the \texttt{IO} monad in the interpreter loop, it was
changed to an implementation based on an unboxed array of size 8.

\subsection*{Code layout}

The code is written in Haskell and comprises a number of files:
\fixme{code description for the other parts}
\begin{description}
\item[Decode.hs] The instruction decoder for the interpreter. Also
  contains the algebraic datatype for the instruction set
\item[Interpreter.hs] The interpreter step-loop and its helpers
\item[Main.hs] Main function and execution
\item[Register.hs] Implementation of the register set
\item[SequenceState.hs] Instances of the \texttt{State} typeclass
  implementing the Store
\item[State.hs] The type class of \texttt{State}, the class of UM stores.
\item[UmixVM] The converter from byte streams to opcode streams.
\end{description}
\subsection*{Program Usage}

To build the program, we use:
\begin{verbatim}
ghc -O2 -fglasgow-exts -XFlexibleInstances -auto-all --make \
  *.hs -o uvm
\end{verbatim}

The optimization parameter \texttt{-O2} is fairly important because we
need the strictness analyzer to kick in. To run the program on an UM
file, we use:
\begin{verbatim}
./uvm data/sandmark.umz
trying to Allocate array of size 0..
trying to Abandon size 0 allocation..
...
\end{verbatim}
which will halt when the UM-program halts.

\paragraph{Performance notes}

We get to the decryption key prompt in $2.26$ seconds run-time on a
$2.4$Ghz Core 2 Duo machine. The sandmark benchmark completes in $13$
minutes on the same machine, which is acceptable performance for
playing with the UM.

%%% Local Variables:
%%% mode: latex
%%% TeX-master: "master"
%%% End:
