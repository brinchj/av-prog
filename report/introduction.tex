\chapter{Introduction}

In this project we conduct the tasks we were given at the beginning of
the course:
\begin{enumerate}
\item Implementation of universal machine specification (UM)
\item Parse a language
\item 2D code generator -- compilation into 2d.
\end{enumerate}
We have decided to implement the solution in Haskell, despite none of
us have great knowledge with the platform. The F\# language was deemed
too young, especially on the Mono platform we would be using. The
advantage of the Haskell GHC compiler is its known stability.

This report comprises 3 main parts corresponding to each of the
above tasks.

We started out trying to parse and implement a subsection of the O'cult
language. In order to understand the O'cult language we began
formalizing the semantics, which proved to be a fruit-full labour, as
it resulted in a twelf formalization, which can be found in
\ref{appendix:twelf}. However, due to problems in parsing the O'cult
language, as well as lack of time, we decided to instead implement the
Simple Expression Language (SEL) proposed at the website, with a few
additions - such as function calls and variables, in order to make it interesting.

We expect the reader to understand the concepts introduced in the
course as well as the basics of Haskell programming. As such, we won't
explain, for example, what a Monad is.

\paragraph{Overview}
In the first part, we report on the UM interpreter we wrote in
Haskell. Then we quickly cover our attempts of compiling O'cult before
we focus on the SEL: We have a section covering parsing, a section
covering compilation and a section with code layout in 2d.

%%% Local Variables: 
%%% mode: latex
%%% TeX-master: "master"
%%% End: 
