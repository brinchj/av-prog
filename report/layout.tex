\section{Layouting 2d}

In this section we describe how we decided to lay out 2d
programs. Recall from the compiler, that we construct a graph of
boxes. This graph is the input to the 2d layouter and we provide one
for each function. Each graph is compiled into a module and these are
gathered together for output.

The layout-process can best be described as a series of phases where
each phase make the representation more explicit. Finally, we have so
much information we can print out the layout, one line at a time.

The input-graph is first made explicit: We note the source and target
of each wire, taking special care of wires that enters or leaves the
module. The graph is then represented as a list of boxes, which we run
topological sort on. The sorted boxes are ordered from left to right
inside the module with the wiring underneath boxes (we call this part
the bus). Effectively this means we save space and can represent any
box configuration allowed for in 2d by linearization.

We then run liveness analysis on the bus. This tells us what wires are
live over which boxes and gives us knowledge of where a wire can run
safely -- without colliding with other wires in invalid ways.

After this, we can output each box-section individually as we have
full knowledge of which wires are entering and leaving the box and
what wires are passed through on the bus below the boxes. An example
of this wiring principle is given in figure \ref{fig:1}
\begin{figure}
  \centering

  \begin{center}
\begin{verbatim}
       &                               &
       &                               &
,.....|&...............................&, line0
:name |&                               &: line1
:     |&   ++                          &: line2
:     |& +-+v                          &: line3
:     |& | *========================*  &: line4
:     |&+#>!Box contents            !-+&: line5
:     |&|| *========================* |&: line6
:     |&||                         +-+|&: line7
:     +&+|                           ||&:
-------&-+                           ||&:
:      &-----------------------------##&:
:      &-----------------------------##&:
:      &                             +#&-
:      &------------------------------#&-
:      &                              +&-
,......&...............................&,
       &                               &
       &                               &
\end{verbatim}
  \end{center}
  \caption{Box layout example}\label{fig:1}
\end{figure}

As can be seen, we handle the entry and exit of the module specially
by crafting two fake boxes that gets special treatment. Each box is
laid out by a seperate function for each line. The functions are fed
with knowledge of which wires are present and the required width of
the box; they produce the correct layout in each case. Rendering the
bus is carried out by a tail-recursive function that keeps track of
when wires crosses.

We can only do it this way because we have made a number of passes
that gives us the exact information we need to draw each line
separately.

\paragraph{Use of language features in the layouter}
\label{sec:use-lang-feat}

Our layouter uses very few special language features. In particular,
no monads or other structures are at work. This is due to time
constraint -- one could easily imagine using a State monad to keep
track of crossing wires. It could have removed a considerable amount
of arguments from the functions.

\paragraph{Alternative layouting methods}
\label{sec:altern-layo-meth}

In hindsigt, the approach of drawing line-by-line is more complex than
drawing directly on a canvas. In the canvas-approach one can represent
the canvas as a mapping $\mathbb{N} \times \mathbb{N} \to
\texttt{Char}$. The canvas is then put inside a State-monad and one
creates a combinator language (a DSL) for plotting on the canvas as
well as querying the current state of the canvas. These combinators
would work on the monad.

Each thing you need to draw is then written in the combinator-language
giving an abstraction of 2d layouting. This approach has 2 distinct
advantages from ours:
\begin{itemize}
\item It is modular. Each drawing primitive can act independently of
  other primitives. In contrast, our approach means you can only draw
  a line when you know everything.
\item It is adaptive/iterative. Assume a set of invariants
  ensuring it is always possible to draw wiring in some valid
  way. Then the drawing primitives can automatically adapt to earlier
  plotting on the canvas. If one wants to draw a '|' where there is
  already a '-', the plotter can automatically plot '\#'.

  The same approach simplifies wire-drawing on the bus. We could
  simply query the canvas for the first place we could safely draw the
  wire and then lay it out in that place. Contrast this with our
  approach where we need a seperate step of liveness analysis and
  afterwards a (rather crude) wire-allocation to make it work.
\end{itemize}
Finally, the approach would be a much better fit for a monad, which we
acknowledge was part of the idea in this course.


%%% Local Variables: 
%%% mode: latex
%%% TeX-master: "master"
%%% End: 
